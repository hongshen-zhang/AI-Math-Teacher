\documentclass[UTF8]{ctexart}
\usepackage{amsmath,amsfonts,amssymb}
\title{文件名}
\author{SolveGPT}
\date{\today}

\begin{document}

\maketitle

<#list notes as note>

<#if note.type == 0>
\section{题目}

<#if note.question?has_content>
\subsection{题目}
${note.question}
</#if>

<#if note.answer?has_content>
\subsection{答案}
${note.answer}
</#if>

<#if note.knowledge?has_content>
\subsection{知识点}
${note.knowledge}
</#if>

</#if>

<#if note.type == 1>
\section{知识点}

<#if note.knowledgeName?has_content>
\subsection{名称}
${note.knowledgeName}
</#if>

<#if note.knowledgeContent?has_content>
\subsection{知识点}
${note.knowledgeContent}
</#if>

</#if>

<#if note.type == 2>
\section{作业批改}

<#if note.name?has_content>
\subsection{名称}
${note.name}
</#if>

<#if note.question?has_content>
\subsection{题目}
${note.question}
</#if>

<#if note.correctAnswer?has_content>
\subsection{正确答案}
${note.correctAnswer}
</#if>

<#if note.knowledge?has_content>
\subsection{知识点}
${note.knowledge}
</#if>

<#if note.answer?has_content>
\subsection{学生答案}
${note.answer}
</#if>

\subsection{是否正确}
<#if note.isTrue == 1>
正确
<#else>
\textcolor{red}{错误}
</#if>

<#if note.errText?has_content>
\subsection{错误}
${note.errText}
</#if>

<#if note.comment?has_content>
\subsection{错误}
${note.comment}
</#if>

</#if>

</#list>

\end{document}
